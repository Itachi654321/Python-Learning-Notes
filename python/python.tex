\documentclass{book}%文档类型

\usepackage[heading=true]{ctex}%引入中文,可选参数貌似是改变默认设置
\usepackage{fancyhdr}%改变页面格式
\pagestyle{fancy}%改变页面格式
\lfoot{}%页脚格式
\usepackage{}
\usepackage{amsmath}%某种数学宏包
\usepackage{titlesec}%改变标题格式
\usepackage{amssymb}%某种数学宏包
\usepackage{geometry}%页面设置
\usepackage{makecell}%表格内换行
\usepackage{pifont}%序号
\geometry{centering}%版心居中
\geometry{top=2cm}
\geometry{bottom=2cm}
\geometry{left=2cm}
\geometry{right=2cm}%边缘宽度

\begin{document}
	
	\setlength{\parindent}{0pt}%取消首行缩进
	\begin{titlepage}%标题页
		\vspace*{\fill}
		
		\begin{center}
		\normalfont%字体设置
		{\huge\bfseries Python}
		%此处需要通过空行来换行
		
		\bigskip
		{\Large\itshape 张容康\\2150931}
	\end{center}
    \vspace{\stretch{3}}
    \end{titlepage}

	\thispagestyle{fancy}
	\newpage
	\let\cleardoublepage\clearpage%清除空白页
	\thispagestyle{empty}%清除页眉、页脚
	\tableofcontents%目录页
	
	\part{写在学前}
	
	\chapter{课程安排}
	
	\section{上课时间、地点、座位}
	时间:每周三晚第10-12节课\\
	地点:南楼413教室(南楼四楼中央教室)、机房\\	
	座位:靠窗组最后一排内座,方便电脑充电\\
	
	\section{成绩组成}
	出勤:上课出勤、上机出勤\\
	作业:网上作业、随堂作业\\
	考试:期中考试、期末考试\\
	平时成绩占20\%。\\
	作业提交、测验网址为http://jsjjc.tongji.edu.cn(同济大学计算机基础教研室)。\\
	
	\section{教材、作业}
	教材:Python程序设计基础(上)\\
	作业:实验作业与补充作业\\
	
	\chapter{资源}
	
	\section{课程讲义}
	Python程序设计基础(上)\\
	
	\section{GoodNotes}
	贝叶斯思维统计建模的Python学习法\\
	从Excel到Python\\
	集体智慧编程\\
	利用Python进行数据分析\\
	零起点Python机器学习快速入门\\
	流畅的Python\\
	深度学习入门\\
	数据科学入门\\
	网络爬虫,Python和数据分析\\
	用Python写网络爬虫\\
	征服Python\\
	A Primer on Scientific Programming with Python\\
	Effective Python\\
	Head First Python\\
	Learning Python\\
	MySQL Cookbook\\
	Natural Language Processing with Python\\
	PyQt5快速开发与实战\\
	Python Cookbook\\
	Python Machine Learning\\
	Python编程:从入门到实践\\
	Python编程(上)\\
	Python编程(下)\\
	Python编程快速上手\\
	Python标准库\\
	Python参考手册\\
	Python高级编程\\
	Python高级编程(1)\\
	Python核心编程\\
	Python灰帽子——黑客与逆向工程师的编程之道\\
	Python机器学习及实践\\
	Python技术手册\\
	Python深度学习\\
	Python网络编程\\
	Python学习手册\\
	Python游戏编程快速上手\\
	Python源码剖析\\
	Python中文官方文档\\
	
	\section{B站}
	花了2万多买的Python教程全套\\	
	
	\chapter{指南、问题、错误、日常}
	
	\section{指南}
	Python的学习需要每天坚持,坚持是进步的催化剂。每天坚持学习Python,不但是知识和能力的积累,还是导致质变的必要条件。\\
	
	\section{问题}
	使用import*导入库中的所有函数和类,会不会存在命名重复的情况并产生覆盖。\\
	如何查看库中含有哪些函数和类。\\
	PyCharm如何安装第三方库。\\
	
	\section{错误}
	GoodNotes-Python编程:从入门到实践-P52-squares.py-\ding{186}:缩进错误。\\

	\section{日常}
	2022.3.16开始学习Python。\\
	\begin{table}[h]
		\begin{tabular}{|c|c|}
			\hline
			2022.3.16&\makecell[c]{做第一次上机实验补充内容\\听晚课\\学习GoodNotes-Python编程:从入门到实践\\学习B站-花了2万多买的Python教程全套}\\
			\hline
			2022.3.17&\makecell[c]{学习GoodNotes-Python编程:从入门到实践\\}\\
			\hline
			2022.3.18&\makecell[c]{学习GoodNotes-Python编程:从入门到实践\\}\\
			\hline
			2022.3.19&\makecell[c]{学习GoodNotes-Python编程:从入门到实践\\}\\
			\hline
			2022.3.20&\makecell[c]{学习GoodNotes-Python编程:从入门到实践\\}\\
			\hline
		\end{tabular}
	\end{table}

\part{课程讲义}

\chapter{Python程序设计基础(上)}

\part{GoodNotes}

\chapter{Python编程:从入门到实践}

\section{变量和简单数据类型}

{\heiti 变量、字符串}\\
为变量赋值的通用格式为:\\
$$\makebox{变量=‘语句’}$$\\
变量名只能包含字母、数字、下划线,不能包含空格,开头只能使用字母或者下划线。\\
~\\
修改字符串的大小写:\\
$$\makebox{变量.title/upper/lower()}$$\\
分别可使字符串首字母大写、全部大写、全部小写。\\
~\\
字符串的拼接:\\
$$\makebox{print('语句1'+变量+'语句2')}$$\\
$$\makebox{变量2='语句1'+变量1+'语句2'}$$\\
~\\

{\heiti 空白}\\
添加空白:在字符串中使用'\textbackslash n'和'\textbackslash t'来产生空行或者制表符。换行和制表符本质上是空字符串,所以需要加上引号。在字符串中换行直接在相应位置使用\textbackslash n即可。\\
~\\
删除空白的格式为:\\
$$\makebox{变量.lstrip/strip/rstrip()}$$\\
即可删除左边、左右、右边的空白。\\
~\\

{\heiti Python之禅}\\
输入imort this,运行即输出Python之禅。\\
~\\

\section{列表简介}

{\heiti 创建列表}\\
创建列表的基本格式为:\\
$$\makebox{列表名称=['元素1','元素2',\dots ,'元素n']}$$\\
当列表元素为空时创建为空列表。\\
~\\

{\heiti 访问、使用列表元素}\\
访问、使用列表元素的基本格式为:\\
$$\makebox{列表名称.[元素位置]}$$\\
其中元素位置可以采取正向也可采取反向,正向第一个元素序号为0。\\
~\\

{\heiti 修改列表元素}\\
修改列表元素的格式为:\\
$$\makebox{列表名称.[]='新元素名称'}$$\\
~\\
列表末尾添加元素的格式为:\\
$$\makebox{列表名称.append('新元素名称')}$$\\
~\\
列表中插入元素的格式为:\\
$$\makebox{列表名称.insert(元素位置,'新元素名称')}$$\\
~\\
根据位置删除列表元素的格式为:\\
$$\makebox{del 列表名称.[元素位置]}$$\\
~\\
根据值删除列表元素的格式为:\\
$$\makebox{列表名称.remove('元素名称'/变量名称)}$$\\
~\\
弹出列表元素的格式为:\\
$$\makebox{列表名称.pop(元素位置)}$$\\
当括号内为空时默认弹出末尾元素。\\
~\\

{\heiti 组织列表}\\

按字母顺序排列:\\
$$\makebox{列表名称.sort()}$$\\
括号中使用reverse=true可按字母顺序反向排列。\\
~\\
反转排列顺序:\\
$$\makebox{列表名称.reverse()}$$\\
~\\
确定列表长度:\\
$$\makebox{len(列表名称)}$$\
~\\

\section{操作列表}

{\heiti 遍历列表}\\
遍历列表的格式为:\\
$$\makebox{for x in 列表名称:}$$\\
其中,x为任意指定的变量名称,注意for语句结束要使用冒号,for语句之后采用适当缩进的代码行为对列表元素采取的操作。\\
~\\

{\heiti range函数、数字列表}\\
创建数字列表:\\
$$\makebox{列表名称=list(range(左闭,右开,步长))}$$\\
~\\
遍历range函数:\\
$$\makebox{for x in range(左闭,右开,步长):}$$\\
x为任意指定的数字列表中的变量名称。\\
~\\
统计数字列表:\\
$$\makebox{min/max/sum(列表名称)}$$\\
依次可对数字列表中的元素求最大值、最小值、和。\\
~\\
列表解析:\\
$$\makebox{列表名称=[含数学符号的表达式 for 任意变量名 in range(左闭,右开,步长)]}$$\\
~\\

{\heiti 切片}\\
选取切片:\\
$$\makebox{列表名称[切片左端:切片右端]}$$\\
省略切片左端,默认从0位置开始;省略切片右端,默认到-1结束;省略切片左端和有段,默认输出整个列表。\\
~\\
遍历切片:\\
$$\makebox{for 变量名称 in 列表名称[起始位置:结束位置]:}$$\\
~\\
复制列表:\\
$$\makebox{复制列表名称=原列表名称[:]}$$\\
这是产生的是两个分别独立的列表。\\
如果采用以下格式:\\
$$\makebox{复制列表名称=原列表名称}$$\\
此时两列表相互关联,实际上为同一列表。\\
~\\

{\heiti 元组}\\
元组与列表的区别是单个元素是否可以改变,其余操作与列表相同。\\
元组的格式为:\\
$$\makebox{元组名称=(变量,'字符串')}$$\\
访问元组元素:\\
$$\makebox{元组名称[元素位置]}$$\\
元组元素不可单独修改,但可为整个元组赋值:\\
$$\makebox{元组名称=(新元素)}$$\\
~\\

{\heiti PEP8格式设置指南}\\
访问地址:https://peps.python.org/pep-0008/。\\

\section{if语句}

{\heiti 条件测试}\\
if后表达式称为条件测试,值为True执行下一级,值为False忽略下一级。\\
检查相等/不相等:\\
$$\makebox{if 变量==/!='字符串'/数值:}$$\\
~\\
比较数值:\\
$$\makebox{if 变量==/!=/>/</>=/<=数值:}$$\\
~\\
and、or:\\
and、or用于条件表达式中判断多个条件:\\
$$\makebox{if '条件表达式1' and/or '条件表达式2':}$$\\
~\\
检查元素是否在列表中:\\
$$\makebox{if 变量/'字符串' in/not in 列表名称:}$$\\
~\\
if语句同for函数一样,需要在句末使用冒号。\\
~\\

{\heiti if-elif-else结构}\\
$\makebox{if 条件测试1:}$\\
$\makebox{\hspace{4em}执行语句1}$\\
$\makebox{elif 条件测试2:}$\\
$\makebox{\hspace{4em}执行语句2}$\\
\dots\\
$\makebox{else:}$\\
$\makebox{\hspace{4em}执行语句n}$\\
~\\

{\heiti if语句处理列表}\\
示例:\\
$\makebox{if 列表名称:}$\\
$\makebox{\hspace{4em}for 变量名称 in 列表名称:}$\\
$\makebox{\hspace{8em}执行语句1}$\\
$\makebox{\hspace{4em}else:}$\\
$\makebox{\hspace{8em}执行语句2}$\\
$\makebox{else:}$\\
$\makebox{\hspace{4em}执行语句3}$\\
第一个if语句用于判断列表是否为空,为空执行else。\\
~\\

\section{字典}
{\heiti 操作字典}\\
创建字典:\\
$$\makebox{字典名称=\{'键名称':'值字符串'/数值,\dots\}}$$\\
~\\
访问字典:\\
$$\makebox{字典名称['键名称']}$$\\
当键所对应的值为数值时,需要使用str()转化为字符串输出。\\
~\\
添加键值对:\\
$$\makebox{字典名称['键名称'='值字符串'/数值]}$$\\
~\\

\part{B站}

\chapter{花了2万多买的Python教程全套}

\section{Python自述}

{\heiti Python学习思路}\\
基础课程(自动化运维、多媒体处理、人工智能应用、自动化办公、Web开发、Python爬虫)$\longrightarrow$科学计算(数据分析、数据处理、量化变易)$\longrightarrow$机器学习(数据挖掘、搜索算法、机器学习算法、推荐算法)$\longrightarrow$深度学习。\\
~\\

{\heiti Python学习和就业方向}\\
(1)Web全栈开发方向:\\
前端开发、数据库管理、后台框架技术$\longrightarrow$Web全栈开发工程师;\\
(2)数据科学方向:\\
数据库管理、数据分析、数据可视化、能够制作数据看板、实现数据指标监控$\longrightarrow$数据产品经理、量化交易、初级BI商业分析师。\\
(3)人工智能方向-机器学习:\\
掌握机器学习常用算法和思想、利用Python建立机器学习模型、对一些应用场景进行智能化$\longrightarrow$数据分析工程师、机器学习算法工程师、搜索算法工程师、推荐算法工程师。\\
(4)人工智能方向-深度学习:\\
掌握深度学习常用框架、自行搭建图像相关算法、自行搭建NLP相关算法、掌握GAN网络相关算法$\longrightarrow$人工智能工程师。\\
~\\

{\heiti Python语言特点}\\
(1)是一种跨平台的计算机程序设计语言;\\
(2)是一种解释性的语言,没有编译的环节;\\
(3)是一种交互式语言;\\
(4)是面向对象语言。\\
~\\

{\heiti Python解释器}\\
解释器是一种计算机程序,可将高级程序语言转换为机器代码。其中:\\
IDLE是Python自带的简单开发环境;\\
Python 3.x(32/64-bit)是交互式命令行程序;\\
Manuals是官方技术文档;\\
Module Docs是已安装的模块文档。\\
~\\

{\heiti Python编译器}\\
编译器可将高级语言编写的程序转换成机器代码,将人可读的代码转换成计算机可读的代码(0和1)。\\

\section{基础语法}
{\heiti 输出函数print}\\
输出函数print可以输出到显示器或者文件中。\\
整数、浮点数、变量、含有运算符的表达式(输出运算结果)可以直接输出,字符串需要加上引号输出。\\
同一个print函数中的内容输出在同一行,不同print函数中的内容会换行输出。\\
输出到指定文件的方式为:\\
$\makebox{fp=open('文件地址','a+')}$\\
$\makebox{print(输出内容,file=fp)}$\\
$\makebox{fp.close()}$\\
如果指定文件位置存在,则在原文件后追加新内容。\\
如果需要输出多个字符串,字符串之间需要用逗号分割,如:\\
$$\makebox{print('字符串1','字符串2')}$$\\
~\\

\part{杂记}

{\heiti 输入函数input}\\
通用格式为:\\
$$\makebox{变量=input('提示语')}$$\\
此时赋予的数据类型均为string型。\\
~\\
进行算数运算时,强制转换数据类型的格式为:
$$\makebox{变量=int/float(input('提示语'))}$$\\
~\\
在提示语中加入变量的方式为:\\
$$\makebox{变量1=input(‘提示语1’+str(变量2)+‘提示语2’)}$$\\
~\\
同时输入多个变量的格式为:\\
$$\makebox{变量1,变量2,\dots ,变量n=input('提示语').split()/(',')}$$
输入时,可以用用空格分离变量值,也可以用逗号分离变量值。此时返回值为一个列表,不能直接转换类型。\\
~\\
用输入函数为多个变量赋予其他数据类型的方法为:\\
$$\makebox{变量1,变量2,\dots ,变量n=map(int/float,input('提示语').split()/(','))}$$\\
~\\

{\heiti 内置函数chr与进制}\\
使用格式为:\\
$$\makebox{变量=chr(进制提示符\quad 相应进制下的ASCII码)}$$\\
0b为二进制,0o为八进制,0d为十进制,0x为十六进制。输入值为相应进制下的ASCII码,返回值为对应的字符。\\
~\\
将十进制下的字符转换为其他进制下的ASCII码方法为:\\
$$\makebox{bin(),oct(),hex()}$$\\
可依次将十进制转换为二进制、八进制、十六进制的ASCII码。\\
~\\
其他进制转换为十进制的方法:\\
$$\makebox{eval('0xASCII码')}$$\\

{\heiti 保留小数位数}\\
(1)$$\makebox{变量=浮点数}$$\\
$$\makebox{round(变量,有效数字位数)}$$\\
~\\
(2)$$\makebox{'\%.nf'\%a}$$\\
其中,n为保留小数位数,f表示数据类型为浮点数。\\
~\\

{\heiti 随机数}\\
随机生成一个[0,1)的浮点数:\\
$$\makebox{random.random()}$$\\
~\\
随机产生一个[a,b)的浮点数:\\
$$\makebox{random.uniform(a,b)}$$\\
~\\
随机生成一个[a,b]的整数:\\
$$\makebox{random.randint(a,b)}$$\\
~\\
从一个序列中随机选取一个元素:\\
$$\makebox{random.choice(sequence)}$$\\
~\\

{\heiti PyCharm中设置Python File开头模板}\\
设置路径为:\\
$$\makebox{File$\rightarrow$Settings$\rightarrow$Editor$\rightarrow$Code Style$\rightarrow$File and Code Templates$\rightarrow$Python Script}$$\\
~\\

\end{document}